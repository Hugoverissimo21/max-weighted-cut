\documentclass[mirror]{revdetua}
%
% Valid options are:
%   shortpaper -------- \part and \tableofcontents not defined (default)
%   portugues --------- main language is Portuguese
%   draft ------------- draft version
%   final ------------- final version (default)
%   times ------------- use times (postscript) fonts for text
%   mirror ------------ prints a mirror image of the paper (with dvips)
%   visiblelabels ----- \SL, \SN, \SP, \EL, \EN, etc. defined
%   invisiblelabels --- \SL, \SN, \SP, \EL, \EN, etc. not defined (default)
%
% Note: the final version should use the times fonts
% Note: the really final version should also use the mirror option
%

\usepackage[portuguese]{babel} % Adjust based on your needs
\usepackage[utf8]{inputenc} % Required for Portuguese
\usepackage{amsmath} 
\usepackage{comment}
%%%%%%%%%%%%%%%%%%%%%%%%%%%%%%%
% compiling:
% Recipe: xelatex
% Recipe: pdflatex -> bibtex -> pdflatex -> pdflatex
% Recipe: xelatex
%%%%%%%%%%%%%%%%%%%%%%%%%%%%%%%
\begin{document}

\Header{0}{09}{Novembro}{2024}{1}
% Note: the month must be in Portuguese

\title{title}
\author{Hugo Campos sss Veríssimo}
\maketitle

\begin{abstract}
abstrato em pt bla bla ns q in English
\end{abstract}

\begin{resumo}
resumo baksdakdjsa in Portuguese
\end{resumo}

\begin{keywords}
keywords Note: in English (optional)
\end{keywords}

\begin{palavraschave}
keywords pt in Portuguese (optional)
\end{palavraschave}

\section{Introdução}

Os problemas em grafos, atualmente, são amplamente estudados, pelo facto de terem a capacidade de modelar diversas situações reais, desde as mais palpáveis, como redes de computadores (problema \textit{Minimum Spanning Tree}) até às mais abstratas, como física teórica (problema \textit{Maximum Weight Cut}) \cite{WP24}.

Este relatório visa explorar o problema \textit{Maximum Weight Cut}, conhecido em português por Corte de Peso Máximo, que consiste na divisão de um grafo não direcionado, $G(V, E)$, onde $|V| = n$ vértices e $|E| = m$ arestas de peso $w_{i,j} \geq 0\ \forall\ (i,j) \in E$, em dois subconjuntos complementares, $S$ e $T$, de forma a maximizar a soma dos pesos das arestas que ligam os dois conjuntos \cite{SC03}, isto é
\begin{equation*}
    \begin{split}
        \max \sum_{i \in S,\ j \in T} w_{i,j} \\ 
        \left\{\begin{split}
            &S \cup T = V \\
            &S \cap T = \emptyset
        \end{split}\right.
    \end{split}
\end{equation*}

Apesar do problema oposto, conhecido como \textit{Minimum Weight Cut}, ter um algoritmo de resolução em tempo polinomial, em certas condições, o problema \textit{Maximum Weight Cut} não o possui, sendo um problema \textit{NP-Hard}. Isto implica que à medida que o tamanho do grafo aumenta, encontrar soluções exatas para este problema tornam-se computacionalmente caras \cite{WP24}.

Ao longo deste relatório, serão abordadas duas estratégias para a resolução do problema: uma busca exaustiva e uma heurística gulosa.

% pode se meter aqui foto la do problema

\section{org e codigo}

ns q gerar graficos, ficheiro no github?

etc etc

\section{Algoritmo de Pesquisa Exaustiva}


a busca exaustiva, que explora todas as combinações possíveis para encontrar uma solução ótima

\section{Algoritmo de Pesquisa Gulosa}

técnica heurística gulosa (greedy), que procura uma solução aproximada de maneira eficiente, tomando decisões locais que parecem melhores no momento, embora sem garantia de que serão as ótimas globais.

\section{qqcena}

jdsauhdisa
aa a a
\section{qqcena}
dasdsad sn q disse

\bibliography{refs}

\end{document}
