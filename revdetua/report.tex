\documentclass[mirror]{revdetua}
%
% Valid options are:
%   portugues --------- main language is Portuguese
%   final ------------- final version (default)
%   times ------------- use times (postscript) fonts for text
%   mirror ------------ prints a mirror image of the paper (with dvips)
%   visiblelabels ----- \SL, \SN, \SP, \EL, \EN, etc. defined
%   invisiblelabels --- \SL, \SN, \SP, \EL, \EN, etc. not defined (default)
%
% Note: the final version should use the times fonts
% Note: the really final version should also use the mirror option
%

\usepackage[portuguese]{babel}
\usepackage[utf8]{inputenc}
\usepackage{amsmath} 
\usepackage{comment}
\usepackage{algorithm}
\usepackage{algpseudocode}
%\floatname{algorithm}{Algoritmo}
%%%%%%%%%%%%%%%%%%%%%%%%%%%%%%%
% compiling:
% Recipe: xelatex
% Recipe: pdflatex -> bibtex -> pdflatex -> pdflatex
% Recipe: xelatex
%%%%%%%%%%%%%%%%%%%%%%%%%%%%%%%
\begin{document}

\Header{01}{01}{Novembro}{2024}{1}

\title{Maximum Weight Cut}
\author{Hugo Veríssimo - 124348 - hugoverissimo@ua.pt}
\maketitle

\begin{abstract}
abstrato em pt bla bla ns q in English  in English in English in English in English in English in English in English in English in English in English in English in English in English
\end{abstract}

\begin{resumo}
Este trabalho apresenta a implementação e comparação de dois métodos para resolver o problema \textit{Max Weight Cut}: um algoritmo exaustivo e uma heurística gulosa. O problema do \textit{Max Weight Cut} consiste em dividir um grafo em dois conjuntos complementares, de forma a que o peso das arestas cortadas seja maximizado. %melhorar e bem melhorado, o de baixo é melhor !
%Este trabalho compara duas abordagens para o problema do Max Weight Cut em grafos: um algoritmo exaustivo e uma heurística gulosa. O algoritmo exaustivo, embora garanta a solução ótima, apresenta alta complexidade computacional, tornando-se inviável para grafos grandes. Em contrapartida, a heurística gulosa, que seleciona iterativamente as arestas com maior peso, oferece soluções rápidas, mas não ótimas. Os experimentos demonstram que, embora a heurística não alcance a optimalidade em todos os casos, ela se destaca em eficiência e qualidade da solução para instâncias maiores, evidenciando um trade-off importante entre precisão e desempenho.
\end{resumo}

\section{Introdução}

Atualmente, os problemas em grafos são amplamente estudados, pelo facto de terem a capacidade de modelar diversas situações reais, desde as mais palpáveis, como redes de computadores (problema \textit{Minimum Spanning Tree}) até às mais abstratas, como física estatística (problema \textit{Maximum Weight Cut}) \cite{WANG10}.

Este relatório visa explorar o problema \textit{Maximum Weight Cut}, conhecido em português por Corte de Peso Máximo, que consiste na divisão de um grafo não direcionado, $G(V, E)$, onde $|V| = n$ vértices e $|E| = m$ arestas de peso $w_{i,j} \geq 0\ \forall\ (i,j) \in E$, em dois subconjuntos complementares, $S$ e $T$, de forma a maximizar a soma dos pesos das arestas que ligam os dois conjuntos \cite{SC03}, isto é
\begin{equation*}
    \begin{split}
        \max \sum_{i \in S,\ j \in T} w_{i,j} \\ 
        \left\{\begin{split}
            &S \cup T = V \\
            &S \cap T = \emptyset
        \end{split}\right.
    \end{split}
\end{equation*}

Apesar do problema oposto, conhecido como \textit{Minimum Weight Cut}, ter um algoritmo de resolução em tempo polinomial, em certas condições, o problema \textit{Maximum Weight Cut} não o possui, sendo um problema \textit{NP-Hard}. Isto implica que à medida que o tamanho do grafo aumenta, encontrar soluções exatas para este problema tornam-se computacionalmente caras \cite{SS23}.

Ao longo deste relatório, serão abordadas duas estratégias para a resolução do problema: uma busca exaustiva e uma heurística gulosa.

% pode se meter aqui foto la do problema

\section{Metodologia da Análise}

Com o intuito de analisar o problema em destaque, foi utilizada a linguagem de programação \textit{Python}, conhecida pela sua simplicidade e pela vasta variedade de bibliotecas, tais como \textit{networkx}, \textit{numpy} e \textit{itertools}, que facilitaram a implementação das estratégias propostas.

A análise desenvolvida pode ser dividida em 2 ficheiros principais, sem desmerecer o uso de ficheiros auxiliares, sendo os primeiros:
\begin{verbatim}
    $ python3 graphs.py
    $ python3 benchmarks.py
\end{verbatim}

O ficheiro \textit{graphs.py} teve como propósito gerar vários grafos aleatórios, tendo em conta a semente 124348, com diferentes número de vértices, número de arestas e peso de arestas, de forma a avaliar o comportamento das estratégias aplicadas em diferentes cenários.
% se quiseres dizer dos 12.5 25 50 75% ...

O ficheiro \textit{benchmarks.py} foi o responsável por executar as estratégias de resolução do problema, nomeadamente a busca exaustiva e a heurística gulosa, para vários grafos gerados, guardando os resultados obtidos, tais como tempo de execução, número de operações básicas e precisão do resultado da heurística gulosa, para posterior análise.

%meter imagem de um dos grafos com nome na fig e isso?

\section{Algoritmo de Pesquisa Exaustiva}

Tendo em conta o algoritmo de pesquisa exaustiva, este visa gerar todas as combinações possíveis de subconjuntos de $V$ e, para cada subconjunto, calcular o peso do corte e comparar com o melhor corte encontrado até ao momento, ou seja, o algoritmo tem duas fases: a geração de todos os subconjuntos possíveis e a avaliação de cada subconjunto. Este estratégia garante a obtenção da solução ótima, no entanto, o seu custo computacional é exponencial, sendo impraticável para grafos de grande dimensão. Este algoritmo pode ser representado em pseudocódigo da seguinte maneira:

% rever pt e tudo mais e ingles e titulo e isso
\begin{algorithm}
    \raggedright
    \textbf{Input:} \textit{G} matriz de adjacencia \\
    \textbf{Output:} s, t, weight cut value \\
    \hrule 
    \caption{Exhaustive Search}
    \begin{algorithmic}[1]
        \State input\_set $\gets$ \{0, 1, \ldots, \text{len(G)} - 1\}
        \State subsets $\gets$ \text{EMPTY LIST}
        \State n $\gets$ \text{LENGTH OF input\_set}
        
        \Comment{Generate all subsets}
        \For{r from 0 to n}
            \For{each S in combinations(input\_set, r)}
            \State Add S to subsets
            \EndFor
        \EndFor
        
        \State best $\gets$ input\_set
        \State weight $\gets$ 0
        
        \Comment{Evaluate each subset}
        \For{each S in subsets}
            \State new\_weight $\gets$ 0
            \For{each i in S}
                \For{each j in input\_set - S}
                    \State new\_weight $\gets$ new\_weight + G[i, j]
                \EndFor
            \EndFor
            
            \If{new\_weight $>$ weight}
            \State best $\gets$ S
            \State weight $\gets$ new\_weight
            \EndIf
        \EndFor
        \State S $\gets$ best
        \State T $\gets$ input\_set - best
        \State \Return S, T, weight
    \end{algorithmic}
\end{algorithm}


Pode-se verificar que as duas fases deste algoritmo apresentam complexidades $O(2^n)$ e $O(2^n \times n^2)$, respetivamente. A primeira devido ao processo de geração de todos os subconjuntos possíveis ($2^n$) e a segunda devido ao processo de percorrer cada subconjunto ($2^n$) e para cada qual percorrer todas as combinações de arestas entre o próprio e o seu complementar (no pior caso, $(n \div 2)^2 \rightarrow n^2$).

Assim, verifica-se que a complexidade deste algoritmo é exponencial com um fator polinomial: $O(2^n \times n^2)$, o que reforça a ideia de que este algoritmo é impraticável para grafos de grande dimensão, daí a necessidade de algoritmos alternativos, como o algoritmo de pesquisa gulosa.


\section{Algoritmo de Pesquisa Gulosa}

Atendendo ao algoritmo de pesquisa gulosa com heurísticas, este segue uma abordagem diferente, uma vez que não garante a obtenção da solução ótima, mas sim uma solução aproximada, em tempo polinomial. Isto acontece devido à natureza do algoritmo, que em cada etapa faz escolhas localmente ótimas, sem considerar o impacto global da escolha, na esperança de alcançar um ótimo global.

Para o desenvolvimento este algoritmo, foi necessária uma análise a diversas regras heurísticas \cite{WANG23}, com o objetivo de determinar a melhor estratégia a seguir. Desta forma, a estratégia escolhida pode ser examinada em detalhe no pseudocódigo apresentado a seguir.

% rever tudo se n mudei e isso e ingles e titulo e isso
\begin{algorithm}
    \raggedright
    \textbf{Input:} \textit{G} matriz de adjacencia \\
    \textbf{Output:} s, t, weight cut value \\
    \hrule 
    \caption{Greedy Heuristic}
    \begin{algorithmic}[1]
    
        \State n $\gets$ \text{len}(G)

        \Comment{Extract edges and their weights}
        \State edges $\gets$ \text{EMPTY LIST}
        \For{i from 0 to n - 1}
            \For{j from i + 1 to n - 1}
                \State weight $\gets G[i, j]$
                \State Add (i, j, weight) to edges
            \EndFor
        \EndFor

        \State Sort edges in descending order by weight

        \Comment{Process each edge}
        \State cut\_weight $\gets$ 0
        \State seen, S, T $\gets$ \text{EMPTY SETS}
        \For{each (u, v, weight) in edges}
            \If{u not in seen and v not in seen}
                \State cut\_weight $\gets$ cut\_weight + weight
                \State Add u to S, add v to T
                \State Update seen with \{u, v\}
            \ElsIf{u in S and v not in seen}
                \State cut\_weight $\gets$ cut\_weight + weight
                \State Add v to T, add v to seen
            \ElsIf{u in T and v not in seen}
                \State cut\_weight $\gets$ cut\_weight + weight
                \State Add v to S, add v to seen
            \ElsIf{v in S and u not in seen}
                \State cut\_weight $\gets$ cut\_weight + weight
                \State Add u to T, add u to seen
            \ElsIf{v in T and u not in seen}
                \State cut\_weight $\gets$ cut\_weight + weight
                \State Add u to S, add u to seen
            \ElsIf{u in S and v in T}
                \State cut\_weight $\gets$ cut\_weight + weight
            \ElsIf{v in S and u in T}
            \State cut\_weight $\gets$ cut\_weight + weight
            \Else
                \Comment{Both u and v have been seen, skip}
            \EndIf
        \EndFor

        \State \Return S, T, cut\_weight
    
    \end{algorithmic}
\end{algorithm}

%\pagebreak
% pq texto estava a ficar antes do alg para ter mais texto idk why cuidado

Pode-se observar que o algortimo tem duas etapas: uma de pré-processamento, onde são extraídas as arestas e os seus pesos e ordenados de forma decrescente, e outra de processamento, onde as arestas são processadas de acordo com as regras heurísticas definidas.

A primeira etapa deste algoritmo é a mais custosa em termos de complexidade, sendo $O(n^2 + m \log m)$. No pior caso, quando o grafo é denso, tem-se $m = 0.5 \cdot n(n-1)$, o que faz com que o termo $m \log m$ se torne dominante, e a complexidade da etapa passe a ser aproximadamente $O(n^2 \log n)$.

A segunda etapa, por sua vez, tem uma complexidade $O(m)$, dado que que percorre todas as arestas do grafo, sendo que quando o grafo é denso, tem-se $O(n^2)$.

Portanto, pode-se concluir que esta abordagem apresenta uma complexidade polinomial com um fator logarítmico adicional: $O(n^2 \log n)$.

\section{Análise dos Resultados}

corrido o ficheiro benchmarks.py, foram obtidos varias medidas de resultados ns q bla bla

\subsection{análise de numero de operacoes}

grafico do numero de ops, corresponde a complexidade referida formalmente anteriormente

GRAFICO

é de notar q n se nota diferencas de ops consoante o numero de arestas, isto pq o os algoritmos acabam por analisar todas as arestas, mesmo q tenham valor 0

\subsection{análise de tempo de execução}

para o calculo dos tempos de execucao, foram feitas pelo menos 3 execucoes para cada grafo tendo sido escolhido o tempo minimo entre elas para ter uma maior coerencia/precisao/whatever. para alem disso foi corrido num macbook ...

GRAFICO

ns q verifica se comportamento identeco ao numero de ops, o heuristico é mais rapido e tal

o exuastivo começa a crescer exponencialmente a partir dos ... vertices, tornando mt lenta a sua execucao

ns q foram ajustadas regressoes para os tempos de execucao, de modo a prever o tempo de execucao para grafos de maior dimensao

heuristica, best fit para tempo

$a \times n^2 \log n [1.85901534e-08]$

NMAE (\%) hugo: 8.376400389731693

and

exuastiva, best fit para tempo

$2^{(n - a)} \times n^2 [24.34775728]$

NMAE (\%) hugo: 2.484738758172613

ns q, boas medidas de erro

é de notas q os tempos de execucao nao sao uma grande medida para determinar a eficiencia dos algoritmos e complexidade e ns q, uma vez que depende de processador para processador. neste caso foram feitos num mac bla bla


\subsection{solucoes e precisao}

ns q foram testasdas $2^n$ trivialmente, e 1 na heuristica, ns q, bla bla

usou se o conjunto de grafos do gset

\url{https://github.com/0816keisuke/max-cut-problem-benchmark} e ns q e dps de fazer a precisao para os meus e para estes, tendo em conta a melhor conhecida .... e ns q mts nós e arestas comparativamente aos q criei e gset sao 69 grafos acho eu...

nivel de confianca de $95\%$ em torno da média $[0.7631, 0.8120]$ para ter 95\% de confiança que a média real está dentro do intervalo $[0.7631, 0.8120]$.


\section{Conclusões}
% ou conclusão

dasdsad sn q disse

\bibliography{refs}

\end{document}
